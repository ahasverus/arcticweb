% cd ~/MEGAsync/webdesign/arcticweb/docs
% pdflatex agenda2016

\documentclass[10pt,letterpaper]{article}

\usepackage[utf8]{inputenc}
\usepackage{xcolor}
\usepackage[top=2.5cm, bottom=2.5cm,left=2.5cm,right=2.5cm]{geometry}
\usepackage{fancyhdr}
\usepackage{graphicx}

\pagestyle{fancy}

\fancyhead[L]{\textbf{ARCTICWEB -- Workshop 11--13 February 2016}}
\fancyfoot[C]{}
\fancyfoot[R]{\textbf{\thepage/3}}

\begin{document}

\section*{Objectives of the Workshop}

\vspace{0.35cm}

\begin{enumerate}
    \item{Prioritize research questions and identify P.I.}
    \item{Identify and set up large-scale experiments over latitudinal gradients}
    \item{Draft rules of functioning and organisation of ArcticWEB network -- Establish a communication strategy}
    \item{Prepare a match plan for further international funding grants applications}
    \item{Draft a synthesis paper on the foundation of ArcticWEB (inspired from Berteaux \textit{et al.} 2010 promoting the CC-Bio project)}
\end{enumerate}

\vspace{0.45cm}

\section*{Short programm of the annual symposium of the Centre for Northern Studies 2016 and the ArcticWEB workshop}

\vspace{0.35cm}

\subsection*{Thursday 11 February 2016 -- Université du Québec à Trois Rivières (UQTR)}

\vspace{0.35cm}

\begin{description}
    \item[08:30] Welcoming address -- Najat Bhiry (CEN director) and Esther Levesque (UQTR)
    \item[09:00] Invited talk -- Alexandre Langlois -- Remote sensing in the Arctic (in French)
    \item[10:20] Invited talk -- Francois Costard -- Thermo-mechanic erosion processes (in French)
    \item[13:15] Plenary Lecture -- Nigel Gilles Yoccoz -- Modéliser la dynamique des écosystèmes arctiques: quelles données pour quels objectifs?(in French)
    \item[16:35] Closing address -- Warwick F. Vincent (CEN scientific director)
    \item[16:30] Poster session -- Posters from ArcticWEB participants
    \item[18:30] Transfer between UQTR and village \textit{Au Chalet en Bois Rond}
\end{description}

The complete program of the symposium (still in French!) is available here\footnote{http://132.203.57.253/document/programme\_courtcolloquecen2016final.pdf}.

\vspace{0.35cm}

\subsection*{Friday 12 February 2016 -- Chalet Aigle Royal, Portneuf\footnote{https://www.auchaletenboisrond.com/chalets-a-louer/aigle-royal}}

\vspace{0.35cm}

\begin{description}
    \item[07:00--09:00] Breakfast -- In each chalet, there will be coffee, tea, hot chocolate, cereals, bread, bacon, eggs and other snacks, please help yourself to make your own breakfast (same on Saturday and Sunday morning)
    \item[09:00--10:00] Plenary -- Open discussion

    Introduction (Joël Bêty and Dominique Gravel)

    Wrap-up on previous ArcticWEB/IPY meetings and directions for the workshop (Pierre Legagneux)

    \item[10:00--10:20] Break

    \item[10:20--12:00] Plenary -- Short presentations (10 min. of presentation max. + 5 min. of questions). These presentations are intended to stimulate exchanges and present \textit{ongoing} projects (future projects or programs will be covered later).

    \begin{description}
        \item[10:20--10:35] Interactions between snow and lemmings (Dorothée Ehrich and Gilles Gauthier)
        \item[10:35--10:50] Why seasonality can matter to population cycles and food web structure? (Frédéric Barraquand)
        \item[10:50--11:05] Integrating seasonality in food web modeling (Isabelle Boulangeat and Pierre Legagneux)
        \item[11:05--11:20] Digging into the arthropod food webs of the High Arctic (Tomas Roslin)
        \item[11:20--11:35] Allochthonous subsidies, a theoretical approach to answer applied questions (Marie-Andrée Giroux, Nicolas Lecomte, Dominique Gravel and Joël Bêty)
        \item[11:35--11:50] Biogeography of food webs and trophic regulation (Dominique Gravel)
    \end{description}

    \item[12:00--13:30] Lunch on site

    \item[13:30--14:15] World café \#1 -- Identify empirical and theoretical questions that could be tackled immediately
    \item[14:15--14:30] Break
    \item[14:30--14:50] Plenary -- Synthesis of World café \#1 (Marie-Andrée Giroux, Pierre Legagneux and Pascale Ropars)
    \item[15:00--18:00] Open discussions and time to draft up a synthesis paper based on all discussions (possibility to work in groups on specific topics or research questions raised during the World café)
    \item[18:00--22:00] Diner on site

\end{description}

\vspace{0.35cm}

\subsection*{Saturday 13 February 2016 -- Chalet Aigle Royal, Portneuf}

\vspace{0.35cm}

\begin{description}

    \item[09:00--09:15] Plenary -- Introduction and wrap-up of diner discussions (Joël Bêty and Dominique Gravel)
    \item[09:15--10:00] World café \#2 -- Two questions: 1) Identify future experiments (or standardized replicated protocols) across a latitudinal gradient, and 2) Identify big questions that we could envision in 5--10 years (3 tables)
    \item[10:00--10:15] Break
    \item[10:15--10:35] Plenary -- Synthesis of World café \#2 (Marie-Andrée Giroux, Nicolas Lecomte, Pierre Legagneux and Pascale Ropars)
    \item[10:35--12:00] Open discussions and time to draft up a synthesis paper based on all discussions (possibility to work in groups on specific topics or research questions raised during the World cafés)
    \item[12:00--13:30] Lunch on site

    \item[13:30--14:15] Plenary -- Short presentations (10 min. of presentation + 10 min. of questions)

    \begin{description}
        \item[13:30--13:45] Why and how fundamental research at the ecosystem level is relevant for governmental agencies? (Grant Gilchrist and Paul Smith) \textbf{to be confirmed}
        \item[13:45--14:00] The Norwegian experience: Climate-ecological Observatory for Arctic Tundra (Dorothée Ehrich) \textbf{to be confirmed}
        \item[14:00--14:15] Census of marine life, Canadian Healthy Oceans Network, Amundsen (ArcticNET), Green edge, Notre Golfe, etc. What can we learn from the experience of existing networks in oceanography? (Philippe Archambault)
    \end{description}

\end{description}

\newpage

\noindent \textbf{Working Groups -- Not a world café here!}

\begin{itemize}
    \item \textbf{Topic \#1} ArcticWEB rules of functioning and organisation -- What can we learn from ASDN or other existing networks? Chair: Gilles Gauthier
    \item \textbf{Topic \#2} ArcticWEB internal communication strategy. Chair: Dominique Gravel
    \item \textbf{Topic \#3} ArcticWEB financial opportunities -- What can be done from each participating countries? What are the international opportunities? Chair: Dominique Berteaux
\end{itemize}

\begin{description}

    \item[14:30--15:30] Synthesis of the 3 Working groups (Dominique Berteaux, Gilles Gauthier and Dominique Gravel)
    \item[15:30--18:00] Open discussions and time to draft up a synthesis paper based on all discussions
    \item[18:00--22:00] Diner on site

\end{description}

\vspace{0.35cm}

\subsection*{Sunday 14 February 2016 -- Chalet Aigle Royal, Portneuf}

\vspace{0.35cm}

\begin{description}
    \item[09:00--11:00] Plenary -- Wrap-up, task assignments, and tentative schedule for future progress
\end{description}

\noindent Small snack on site and departure of all participants -- Transfer to Québec and Montréal Airports for international attendees. Departure time: 11:30.

\vspace{2cm}

\noindent{}Ce projet est réalisé grâce à l’appui financier reçu du Secrétariat aux affaires intergouvernementales canadiennes du gouvernement du Québec\footnote{http://www.saic.gouv.qc.ca/index.asp} en vertu des programmes de soutien financier en matière de francophonie canadienne.

\noindent{}\includegraphics[scale=0.05]{logo-saic}

\end{document}
